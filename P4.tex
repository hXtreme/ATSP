\documentclass[./main.tex]{subfiles}

\begin{document}
	\begin{definition}
		%Fix some $\delta$. A set $S\in \lam$ is called \textbf{reducible} if
		Fix some $\epsilon$. A set $S\in \lam$ is called \textbf{reducible} if
		\[
			%D_{\max}(S) \leqslant \delta \val_\calI(S),
			D_{\max}(S) \leqslant (1 - \epsilon)\cdot \val_\calI(S),
		\]
		and \textbf{irreducible} otherwise.
		An instance $\calI$ is irreducible if every laminar set is irreducible.
	\end{definition}

	%By this definition, we see that if we contract a reducible set $S$, the LP-value goes down by at least $\val_\calI(S) - D_{\max}(S) \geqslant(1-\delta)\cdot \val_\calI(S)$.\\
	By this definition, we see that if we contract a reducible set $S$, the LP-value goes down by at least $\val_\calI(S) - D_{\max}(S) \geqslant \epsilon\cdot \val_\calI(S)$.\\

	\begin{algorithm}[h]\label{alg:1}
		\caption{LAMINARLY WEIGHTED $\rightarrow$ IRREDUCIBLE}
		\SetKwFunction{Lift}{Lift}
		\eIf{$\calI$ is irreducible}{
			\Return $A_{irr}(\calI)$
		}{
			$S \leftarrow$ any minimal reducible set of $\lam$.\\
			$T \leftarrow$ tour in $\calI/S$ found using recursion.\\
			$T_S\leftarrow$ tour in $\calI[S]$ found using $A_{irr}$.\\
			$T' \leftarrow \Lift(T)$, $T_S' \leftarrow \Lift(T_S)$.\\
			\Return $T'\cup T_S'$
		}
	\end{algorithm}

	\begin{theorem}
		%A $\rho$-approximation algorithm for ATSP on irreducible instances $(A_{irr})$ implies a $\frac{2\rho}{1-\delta}$-approximation algorithm for ATSP on laminarly weighted instances.
		A $\rho$-approximation algorithm for ATSP on irreducible instances $(A_{irr})$ implies a $\frac{2\rho}{\epsilon}$-approximation algorithm for ATSP on laminarly weighted instances.
	\end{theorem}
	\begin{proof}
		Note that step 6 of the algorithm is valid because $\calI[S]$ is irreducible - all the laminar sets inside $S$ were already irreducible, and all singleton sets are trivially irreducible, so the new laminar set is also irreducible.\\
		Also note that the base case (i.e $\calI$ is already irreducible) is trivial.

		For the sake of induction, assume that our algorithm gives us a tour $T$ of $\calI/S$ such that:
		\[
			%w_{\calI/S}(T) \leqslant \frac{2\rho}{1-\delta}\cdot \val(\calI/S).
			w_{\calI/S}(T) \leqslant \frac{2\rho}{\epsilon}\cdot \val(\calI/S).
		\]
		We know that $w_{\calI[S]}(T_S)\leqslant \rho\cdot \val(\calI[S]) = 2\rho\cdot\val_S(\calI)$.\\
		Now,
		\begin{align*}
				%ALG &= w_\calI(T') + w_\calI(T_S') \leqslant w_{\calI[S]}(T) + w_{\calI/S}(T_S) \leqslant \frac{2\rho}{1-\delta}\val(I/S) + \rho\cdot \val(\calI[S])\\
				ALG &= w_\calI(T') + w_\calI(T_S') \leqslant w_{\calI[S]}(T) + w_{\calI/S}(T_S) \leqslant \frac{2\rho}{\epsilon}\val(I/S) + \rho\cdot \val(\calI[S])\\
				%&\leqslant \frac{2\rho}{1-\delta}(\val(\calI) - (1-\delta) \val_\calI(S)) + 2\rho\cdot\val_\calI(S)\\
				&\leqslant \frac{2\rho}{\epsilon}(\val(\calI) - \epsilon \val_\calI(S)) + 2\rho\cdot\val_\calI(S)\\
				%ALG &= \frac{2\rho}{1-\delta} \val(\calI)
				ALG &= \frac{2\rho}{\epsilon} \val(\calI)
		\end{align*}
	\end{proof}
\end{document}
