\begin{document}
Consider any $S\in L$. Let $S_1,\ldots, S_k$ be the strongly connected components of $S$ in topologically sorted order. We know that
\begin{enumerate}
	\item $x(\delta^-(S_1)) \geq 1$
	\item $x(\delta^-(S)) = 1$
	\item Every edge into $S_1$ is also an edge into $S$
\end{enumerate}
It follows that $x(\delta^-(S_1)) = 1$. Using induction along with a similar argument, one can also see that for all $i$, $x(\delta^-(S_i)) = x(\delta^+(S_i)) = 1$, and the only edges from one SCC to another are edges† from $S_i$ to $S_{i + 1}$. As a consequence, there always exists a path from $u\in S_{in}$ to $v\in S_{out}$ within $S$. This structure allows us to "shortcut" paths that enter or exit $S$ multiple times - we can replace the entire portion of the path between the first time it enters $S$ and the last time it exits $S$ with a path completely inside $S$.\vspace{2mm}
\begin{figure}[h]
	\centering
	\begin{tikzpicture}[.->, >=stealth', auto, node distance = 2cm,on grid, semithick]

		% \node[state]		(name)	[location]			{label};
		\node[rectangle]	(s0)						{};
		\node[state]		(s1)	[right=2cm of s0]	{$S_1$};
		\node[state]		(s2)	[right=1.5cm of s1]	{$S_2$};
		\node[rectangle]	(sd)	[right=1.5cm of s2]	{$\cdots$};
		\node[state]		(sk)	[right=1.5cm of sd]	{$S_k$};
		\node[rectangle]	(so)	[right=2cm of sk]	{};

		% Set S
		\draw (4.25,0) ellipse (3cm and 1cm);
		% Little bit of a hack
		\node[rectangle] (S) [above right=0.75cm and 1cm of s2] {$S$};

		\path
		%(source)	edge		node	{TEXT}	(dest)
		(s0)		edge		node	{}		(s1)
		(s1)		edge		node	{}		(s2)
		(s2)		edge		node	{}		(sd)
		(sd)		edge		node	{}		(sk)
		(sk)		edge		node	{}		(so);
	\end{tikzpicture}
	\caption{Paths traverse through a tight set $S$ by entering at $S_1$, visiting each connected component in $S$ exactly once before leaving from $S_l$.}
\end{figure}\vspace{4mm}
\begin{definition} Let $\bm{\textbf{value}_I(S)}$ be the LP value of $I$ stored within $S$, and let $\bm{\textbf{value}(I)}$ be the total LP value of $I$. Formally, $$\bm{\textbf{value}_I(S)}:=\sum\limits_{R\in L:R\subsetneq S}2\cdot y_R\qquad \bm{\textbf{value}(I)}:=\val_I(V)$$
\end{definition}\vspace{2mm}
\begin{definition} Let $\bm{d_S(u,v)}$ be the weight of the shortest path from $u$ to $v$ inside $S$. Also, let \begin{equation}\bm{D_S(u,v)}:=\sum_{R\in L : u\in R\subsetneq S}y_R + d_s(u,v) + \sum_{R\in L : v\in R\subsetneq S}y_R\label{def-1}\end{equation}
	Furthermore, let $D_{\max}(S) = \max_{u \in S_{in}, v \in S_{out}}D_S(u,v)$.
\end{definition}
\subfile{Diagrams/D.tex}
Intuitively, $D_s(u,v)$ is the total cost of entering $u$ from outside $S$, a shortest path from $u$ to $v$, and exiting from $v$ to outside $S$.See \hyperref[fig:Duv]{Figure \ref{fig:Duv}}.\vspace{2mm}
\begin{lemma}\label{lemm:2:D-val} $S\subseteq V$ be any set such that $L\cup \{S\}$ is a laminar family. Then\begin{enumerate}
	\item[(1)] $d_S(u,v)\leq \val_I(S)$ if there exists a $u-v$ path inside $S$.
	\item[(2)] $D_S(u,v)\leq \val_I(S)$ if $u\in S_{in}$ or $v\in S_{out}$
	\end{enumerate}
\end{lemma}
\begin{proof}
	The main idea for the proof for $(1)$ is that a path that enters some laminar set multiple times can be shortcut to just enter that laminar set once. After repeatedly shortcutting until each laminar set is entered at most once, the cost of the path that each set $R$ is responsible for is at most $2\cdot y_R$ (one for entering $R$, and one for exiting $R$), and $(1)$ follows from the definition of $\val_I(S)$.\vspace{2mm}
	\\The idea for $(2)$ is similar. One can show (using a case-by-case analysis) that for all $R\in L$ with $R\subsetneq S$, after shortcutting the path, at most two of the following are true:
	\begin{enumerate}
		\item $u\in R$
		\item $v\in R$
		\item The path enters $R$
		\item The path exits $R$
	\end{enumerate}
	In other words, every laminar set $R\subsetneq S$ appears at most twice in the right hand side of Equation \ref{def-1}. $(2)$ follows from the definition of $\val_I(S)$.
\end{proof}\vspace{2mm}
This gives us useful bounds on costs our algorithm will be incurring.
\end{document}
