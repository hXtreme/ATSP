\begin{document}
	Consider any $S\in \lam$. Let $S_1,\dots, S_k$ be the strongly connected components of $S$ in topological order.
	We know that
	\begin{enumerate}
		\item $x(\delta^-(S_1)) \geqslant 1$; as $x$ is a feasible solution to $\LP(G,w)$,
		\item $x(\delta^-(S)) = 1$; as the primal constraint for $S$ is tight,
		\item Every edge into $S_1$ is also an edge into $S$.
	\end{enumerate}
	It follows that $x(\delta^-(S_1)) = 1$, that is, all of the incoming flow into $S$ enters through $S_1$. As a result, the only edges into $S_2$ are from $S_1$. Since $x(\delta^-(S_2))\geqslant 1$, all of the outgoing edges from $S_1$ must go to $S_2$, and $x(\delta^-(S_2)) = 1$.
	\\Extending this argument inductively, one can also see that for all $i$, $x(\delta^-(S_i)) = x(\delta^+(S_i)) = 1$, and $S_{in}\subseteq S_1,S_{out}\subseteq S_k$.
	As a consequence, there is always a path inside $S$ from $u\in S_{in}$ to $v\in S_{out}$, and every path that goes through $S$ visits all its connected components.

	This structure allows us to "shortcut" paths that enter or exit $S$ multiple times. Consider a path $P$ that enters $S$ for the first time at $u_1$ and last exits $S$ from $v_k$ then, we can replace the entire portion of $P$ between $u_1$ and $v_k$ with a $u_1 - v_k$ path completely inside $S$.\\
	\subfile{Diagrams/TightPaths.tex}

	\begin{lemma}\label{lemm:2:SiAgree}
		For any strongly connected component $S_i \in S$, $\lam \cup \{S_i\}$ is also a laminar familiy.
	\end{lemma}
	\begin{proof}
		Assume for contradiction that $\lam \cup \{S_i\}$ is not a laminar family.
		Therefore, there must be some $R \in \lam$ such that $R\setminus S_i,\ S\setminus R$, and $R \cap S_i$ are all non-empty.
		We partition $R$ into three sets $R_{<i}, R_i,$ and $R_{>i}$ containing the part of $R$ that intersects with any SCC before $S_i$, with $S_i$, and after $S_i$ respectively. As per our assumption, at least one of $R_{<i},\ R_{>i}$ are nonempty. WLOG, assume that $R_{<i}\neq \emptyset$.
		\begin{align*}
			x(\delta^-(R)) \geqslant x(\delta^-(R_{<i})) + x(\delta(S_i\setminus R_i, R_i)) \geqslant 1 + x(\delta(S_i\setminus R_i, R_i))
		\end{align*}
		Since $S_i$ is strongly connected, there must be at least one edge in $\delta(S_i\setminus R_i, R_i)$, implying $x(\delta^-(R)) > 1$, a contradiction.
	\end{proof}

	\begin{definition}\label{def:value}
		Let $\bm{\textbf{\val}_\calI(S)}$ be the LP value of $\calI$ stored within $S$, and let $\bm{\textbf{\val}(\calI)}$ be the total LP value of $\calI$. Formally,
		\[
			\bm{\textbf{\val}_\calI(S)}:=\sum\limits_{R\in \lam:R\subsetneq S}2\cdot y_R\qquad \bm{\textbf{\val}(\calI)}:=\val_\calI(V)
		\]
	\end{definition}

	\begin{definition}
	\label{def:D}
		Let $\bm{d_S(u,v)}$ be the weight of the shortest path from $u$ to $v$ inside $S$. Also, let
		\begin{equation*}
			\bm{D_S(u,v)}:=\overbrace{\sum_{\substack{R\in \lam \\ u\in R\subsetneq S}}y_R}^{(a)} + \overbrace{\rule{0pt}{4mm}d_s(u,v)}^{(b)} + \overbrace{\sum_{\substack{R\in \lam \\ v\in R\subsetneq S}}y_R}^{(c)}\label{def:1}
		\end{equation*}
		Furthermore, let $\bm{D_{\max}(S)} := \max\limits_{u \in S_{in}, v \in S_{out}}D_S(u,v)$.

		Intuitively, $D_s(u,v)$ is the cost of entering $u$ from outside $S$, taking a shortest path from $u$ to $v$, and exiting from $v$ to outside $S$. See \hyperref[fig:Duv]{Figure \ref{fig:Duv}} for a visualization of this split of cost.\\
	\end{definition}

	\begin{lemma}\label{lemm:2:D-val}
		$S\subseteq V$ be any set such that $\lam \cup \{S\}$ is a laminar family.
		Then,
		\begin{enumerate}
			\item[(1)] $d_S(u,v)\leq \val_\calI(S)$ if there exists a $u-v$ path inside $S$.
			\item[(2)] $D_S(u,v)\leq \val_\calI(S)$ if $u\in S_{in}$ or $v\in S_{out}$
		\end{enumerate}
	\end{lemma}
	\begin{proof}
		The main idea for the proof for $(1)$ is that a path that enters some laminar set multiple times can be shortcut to just enter that laminar set once.
		After repeatedly shortcutting any $u-v$ path until each laminar set is entered at most once, the cost of the path that each set $R$ is responsible for is at most $2\cdot y_R$ (one for entering $R$, and one for exiting $R$), and \textit{(1)} follows from the \hyperref[def:value]{definition of $\val_\calI(S)$}.\\
		The idea for \textit{(2)} is similar. One can show (using a case-by-case analysis) that for all $R\in \lam$ with $R\subsetneq S$, after shortcutting a path, at most two of the following are true:
		\begin{enumerate}
			\item $u\in R$ (contributes to \hyperref[def:D]{$(a)$})
			\item $v\in R$ (contributes to \hyperref[def:D]{$(c)$})
			\item The path enters $R$ (contributes to \hyperref[def:D]{$(b)$})
			\item The path exits $R$ (contributes to \hyperref[def:D]{$(b)$})
		\end{enumerate}
		In other words, every laminar set $R\subsetneq S$ contibutes at most $2\cdot y_R$ in the right hand side of \hyperref[def:1]{Equation \ref{def:1}}, and \textit{(2)} follows from the definition of $\val_\calI(S)$.
	\end{proof}
	\subfile{Diagrams/D.tex}
	This lemma gives us useful bounds on costs the algorithms will be incurring.
\end{document}
