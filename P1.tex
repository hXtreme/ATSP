\documentclass[./main.tex]{subfiles}

\begin{document}
	\begin{definition}
		Given a graph, let $\bm{\delta(U,V)}$ be the set of all edges from $U$ to $V$.
		Let $\bm{\delta^+(U)}$ and $\bm{\delta^-(V)}$ be the set of outgoing and incoming edges from $U$ respectively.
		Let $\bm{\delta(U)}:=\delta^+(U)\cup \delta^-(U)$.
		\\Also for any set $S$, let $\bm{S_{in}}\subseteq S$ be the set of entry points to $S$, and $\bm{S_{out}}\subseteq S$ be the set of exit points of $S$.
	\end{definition}\vspace{4mm}

	\begin{definition}
		Given an edge-weighted ATSP instance (G,w), consider the following LPs:\\
		\begin{minipage}{0.40\textwidth}
			\begin{align*}
				&\textbf{\LP(G,w):}\\
				\min\quad\sum_{e\in E}&w_e\cdot x_e\\
				s.t.\quad x(\delta^+(v)) & = x(\delta^-(v))\quad & v\in V\quad                     \\
				x(\delta(S))             & \geqslant 2\quad           & \emptyset\neq S\subseteq V\quad \\
				x&\geqslant 0
			\end{align*}
		\end{minipage}
		\hfill\vline
		\begin{minipage}{0.65\textwidth}
			\begin{align*}
				\textbf{\DUAL}                                                 & \textbf{(G,w):}             \\
				\max\quad\sum_{S\subseteq V}                                  & 2\cdot y_S                  \\
				s.t.\sum_{S:(u,v)\in \delta(S)}y_S + \alpha_u - \alpha_v & \leqslant w(u,v)\quad (u,v)\in E \\
				y                                                             & \geqslant 0
			\end{align*}
		\end{minipage}
		\vspace{7mm}
	\end{definition}

	\begin{definition}
		$(G,\lam,x,y)$ is called a \textbf{laminarly weighted instance} if $\lam$ forms a laminar family of subsets of $V$ and the following hold:
		\begin{enumerate}
			\item $x$ is a solution to $\LP(G,0)$ and $x > 0$
			\item $y$ is a solution to $\DUAL(G,0)$ and $\lam$ is the support of $y$.
			%\item $y$ is a solution to $\DUAL(G,0)$, and $y_S>0\iff S\in \lam$
			\item The primal constraint corresponding to each $S\in \lam$ is tight
		\end{enumerate}
		Sets in $\lam$ are called laminar sets.
	\end{definition}

	If $w$ is the weight function induced by $y$ (i.e the weight of an edge equals the total $y$-value of the laminar sets it crosses), and if we set $\alpha$ to $0$, then complementary slackness tells us that $x$ and $(y,\alpha)$ are optimal solutions to $\LP(G,w)$ and $\DUAL(G,w)$ respectively.\\

	\begin{claim*}
		Given a graph $G$, we can find an optimal dual solution $(\alpha, y)$ such that the support of $y$ forms a laminar family.
	\end{claim*}

	This can be shown using an "uncrossing" argument - given a dual solution, if the support of $y$ is not laminar, then there must exist two sets in its support, say $A$ and $B$, such that $A\setminus B$ and $B\setminus A$ are nonempty.
	WLOG, assume that $y_A \leqslant y_B$. Then $y_A$ amount of value can be redistributed from $A$ and $B$ to $A\setminus B$ and $B\setminus A$ without disturbing feasibility or optimality.
	Note that $A$ will no longer be in the support of $y$, so $A$ and $B$ are now "uncrossed".
	Applying this procedure repeatedly (and choosing $A$ and $B$ carefully), we can obtain a solution with laminar support in polynomial time.\\
		%\begin{minipage}{\textwidth}
		\begin{theorem}
			An $\alpha$-approximation algorithm for ATSP on laminarly weighted instances implies an $\alpha$-approximation algorithm for ATSP on general instances.
		\end{theorem}
		\begin{proof}
			We are given $I=(G,w)$ as input to general ATSP. Let $x$ be a optimal solution to $\LP(G,w)$ and $(y,\alpha)$ be the laminar optimal solution to $\DUAL(G,w)$ guaranteed by the above claim.
			Let $E'$ be the support of $x$, and $\lam$ be the support of $y$. Then complimentary slackness implies that $\calI'=((V,E'), \lam, x, y)$ is a laminarly weighted instance.\\
			Note that (by complimentary slackness) the induced weight on $\calI'$ is $w'(u,v) = w(u,v) + \alpha_v - \alpha_u$. Although these weights are different from our original weights, the weight of every cycle in $\calI'$ is the same as its weight in $I$ and therefore the weight of every circulation in $\calI'$ is also the same as its weight in $\calI$. In particular, the weight of a tour remains the same, and the optimal \LP-value remains the same.
			This leads to the final algorithm - simply run the $\alpha$-approximation algorithm for laminarly weighted instances on $\calI'$ and return the resulting tour, say $T$.
			Then the weight of $T$ in $\calI$ is given by:
			\[
				w(T) = w'(T)\leqslant \alpha\cdot \val(\calI') = \alpha\cdot\val(I)
			\]
			We have now reduced ATSP to laminarly weighted instances.
		\end{proof}

\end{document}
