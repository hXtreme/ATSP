\documentclass[./main.tex]{subfiles}
%% Macros
\newcommand{\calI}{\ensuremath{\mathcal{I}}}
\newcommand{\calL}{\ensuremath{\mathcal{L}}}

\begin{document}
	\subsection{Definitions}
	\begin{definition}
		We say that a subtour $B$, is a backbone of the instance $\calI$ if for every non-singleton set $S \in \calL$. $B$ crosses $S$ i.e. $\delta(S) \cap B \ne \emptyset$.\\
		The pair $(\calI, B)$ is called the vertebrate pair.
	\end{definition}\vspace{2mm}
	\begin{definition}
		For an instance $\calI$ we say that a subtour $B$ a quasi-backbone if 
		\begin{equation}\label{def:6:q-bb}
			2\sum_{S \in \calL^*} y_S \leqslant \left(1 - \delta\right)\val(\calI)
		\end{equation}
		where $\calL^*$ is the set of all $S \in \calL$ that are not crossed by $B$.
	\end{definition}\vspace{2mm}
	Here we note that a quasi-backbone may not be a backbone and vice-versa especially if majority of $\val(\calI)$ is concentrated on singleton laminar sets in $\calL$.\footnote{While the backbones we construct will also be quasi-backbones but our future reductions do not use this fact.}
	
	\subsection{Finding a Quasi-Backbone}
	\begin{lemma}\label{lemm:6:q-bb}
		Given an irreducible instance $\calI$, we can construct a quasi-backbone $B$ such that it crosses all maximal non-singleton sets in $\calL$ and $w(B) \leqslant (\alpha_{NW} + 3)\val(\calI)$.
	\end{lemma}
	\begin{proof}
		Let $\calL_{\max}$ be the set of all maximal sets in $\calL$ and $\calI'$ be the instance obtained by contracting each set in $\calL_{\max}$. The instance $\calI'$ is node weighted and to find a subtour $B'$ in $\calI$ we find a tour $T$ in $\calI'$ using $\alpha_{NW}$ approx algorithm for Node-Weighted ATSP and lift it. From \hyperref[lemm:3:lift-c]{Lemma \ref{lemm:3:lift-c}} and \hyperref[fact:3:val-c]{Fact \ref{fact:3:val-c}} we know that $w_\calI(B') \leqslant w_{\calI'}(T) \leqslant \alpha_{NW}\val(\calI') \leqslant \alpha_{NW}\val(\calI)$ and that $B'$ crosses all maximal non-singleton laminar sets but, we still need to ensure that it crosses at least $\delta\val(\calI)$ of sets. For this we modify $B'$ to get our quasi-backbone $B$ as follows:
		\begin{itemize}
			\item[-] Let $u^S, v^S$ be the first entry and exit vertex for $S$ in $B'$ respectively and, $u^S_{\max}, v^S_{\max}$ be the vertices corresponding to $D_{\max}(S)$.
			\item[-] Replace the $u^S - v^S$ path by the path $P$ composed of the shortest paths\footnote{These paths exists as $u^S, u^S_{\max} \in S_1$ and $v^S, v^S_{\max} \in S_l$.} from $u^S$ to $u^S_{\max}$ to $v^S_{\max}$ to $v^S$, from \hyperref[lemm:2:D-val]{Lemma \ref{lemm:2:D-val}} $w(P) \leqslant 3\val_{\calI}(S)$
            %\footnote{I believe that with better accounting, we can bring the cost of modification down from $3\val_{\calI}(S)$ to something close to $2\val_{\calI}(S)$.}.
		\end{itemize}
		As $S$ is irreducible, with this modification we automatically satisfy our last requirement while only increasing the weight by at most $3\val(\calI)$.
	\end{proof}
	
	\subsection{Obtaining a Vertebrate Pair}
	Now we would like to find a way to convert a quasi-backbone into a backbone and then use it, along with an approximation algorithm for ATSP on vertebrate pair to approximate ATSP  on irreducible instances.
	
	\begin{algorithm}\label{alg:6:irr-atsp}
		\caption{Irreducible ATSP ($A_{irr}$)}
		\KwIn{
			Irreducible instance $\calI = (G, \calL, x, y)$\\
			Algorithm $A$ for ATSP on vertebrate pair that returns a tour of cost at most $\kappa\val(\calI') + \eta w(B)$.
		}
		Use \hyperref[lemm:6:q-bb]{Lemma \ref{lemm:6:q-bb}} to obtain a quasi-backbone $B$.\\
		$L^*_{\max} \leftarrow $ all non-singleton maximal sets in $\calL^*$.\\
		\For{each $S \in \calL^*_{\max}$} {
			Find $T_S \leftarrow A_{irr}(\calI[S])$.\\
			Use $T_S$ to find $F_S$ for which $S$ is contractible. 
		}
		$\calI' = $ Instance obtained by contracting every $S \in \calL^*_{\max}$.\\
		$T' \leftarrow A(\calI', B)$ and $T \leftarrow $ lift of $T'$.\\
		\Return $T \cup \left(\cup_{S \in \calL^*_{\max}} F_S\right)$
	\end{algorithm}
	
	\begin{theorem}
		Given an irreducible instance $\calI$, \hyperref[alg:6:irr-atsp]{Algorithm \ref{alg:6:irr-atsp}} returns a Tour of $\calI$ of cost no more than $\rho\val(\calI)$ where, $\rho = \frac{\kappa + \eta(\alpha_{NW} + 3)}{1 - 2(1 - \delta)}$.
	\end{theorem}
	\begin{proof}
		\begin{align*}
			w(F) &\leqslant w(T') \leqslant \kappa\val(\calI) + \eta(\alpha_{NW} + 3)\val(\calI) \\
			w(F) &\leqslant \left(\kappa + \eta(\alpha_{NW} + 3)\right)\val(\calI)
		\end{align*}
		\begin{align*}
			w(F_S) &\leqslant w_{\calI[S]}(T_S) \leqslant \rho \val(\calI[S])\\
			w(F_S)&\leqslant 2\rho\val_{\calI}(S)\\
			\cup_{S \in \calL^*_{\max}} w(F_S) &\leqslant 2\rho \left(1 - \delta\right) \val(\calI)  \hspace*{2cm} -  \text{ As all $S$ are disjoint and, \hyperref[def:6:q-bb]{from \ref{def:6:q-bb}} as $B$ was a quasi-backbone.}
		\end{align*}
		\begin{align*}
			Cost(\calI) &= w(F) + \cup_{S \in \calL^*_{\max}} w(F_S) &\\
			&\leqslant \left(\kappa + \eta(\alpha_{NW} + 3)\right)\val(\calI) + 2\rho \left(1 - \delta\right) \val(\calI) & \text{ Substituting $\rho$ and $\delta = 0.75$.}\\
			Cost(\calI) &\leqslant \rho\val(\calI)
		\end{align*}
		
	\end{proof}
\end{document}