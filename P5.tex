\documentclass[./main.tex]{subfiles}

\begin{document}
	\subsection{Definitions}

	\begin{definition}
		We say that a subtour $B$ is a backbone of the instance $\calI$ if for every non-singleton set $S \in \lam$. $B$ crosses $S$ i.e. $\delta(S) \cap B \ne \emptyset$.\\
		The pair $(\calI, B)$ is called the vertebrate pair.\\
	\end{definition}

	\begin{definition}
		We say that a subtour $B$ a quasi-backbone of the instance $\calI$ if
		\begin{equation}\label{def:6:q-bb}
			%2\sum_{S \in \lam^*} y_S \leqslant \left(1 - \delta\right)\val(\calI)
			2\sum_{S \in \lam^*} y_S \leqslant \epsilon\cdot\val(\calI)
		\end{equation}
		where $\lam^*$ is the set of all $S \in \lam$ that are not crossed by $B$.
	\end{definition}

	Here we note that a quasi-backbone may not be a backbone and vice-versa especially if majority of $\val(\calI)$ is concentrated on singleton laminar sets in $\lam$.\footnote{While the backbones we construct will also be quasi-backbones but our future reductions do not use this fact.}

	\subsection{Finding a Quasi-Backbone}

	\begin{lemma}\label{lemm:6:q-bb}
		Given an irreducible instance $\calI$, we can construct a quasi-backbone $B$ that crosses all maximal non-singleton sets in $\lam$ and $w(B) \leqslant (\alpha_{NW} + 3)\val(\calI)$.
	\end{lemma}
	\begin{proof}
		Let $\lam_{max}$ be the set of all maximal sets in $\lam$. If we can find a subtour that crosses each set in $\lam_{max}$, and a $(1-\epsilon)$ fraction of laminar sets inside each $S\in L_{max}$, then that subtour will be a quasi-backbone.\vspace{2mm}
		\\Let $\calI'$ be the instance obtained by contracting each set in $\lam_{max}$. The instance $\calI'$ is node weighted. So we can use the $\alpha_{NW}$-appromxation algorithm for node weighted instances to find a tour of $\calI'$, and lift it to find a subtour $B'$ of $\calI$. This ensures that we cross every maximal set. Now we need to ensure we cross a large fraction of laminar sets inside each $S\in L_{max}$.
		For this we modify $B'$ to get our quasi-backbone $B$ as follows:
		\begin{itemize}[-]
			\item Let $u^S, v^S$ be the first entry and exit vertex for $S$ in $B'$ respectively, and $u^S_{\max}, v^S_{\max}$ be the vertices corresponding to $D_{\max}(S)$.
			\item Replace the $u^S - v^S$ path by the path $P$ composed of the shortest paths\footnote{These paths exists as $u^S, u^S_{\max} \in S_1$ and $v^S, v^S_{\max} \in S_l$.} from $u^S$ to $u^S_{\max}$ to $v^S_{\max}$ to $v^S$, from \hyperref[lemm:2:D-val]{Lemma \ref{lemm:2:D-val}} $w(P) \leqslant 3\cdot\val_{\calI}(S)$
		\end{itemize}
		Since $S$ is irreducible (and there is always a shortest path crossing every laminar set at most twice), this modification automatically satisies our last requirement while only increasing the weight by at most $3\cdot\val(\calI)$.
	\end{proof}

	\subsection{Obtaining a Vertebrate Pair}
	Now we would like to find a way to convert an instance $\calI$ and quasi-backbone $B$ into a vertebrate pair $(\calI', B')$ and then use an algorithm for vertebrate pairs to make some progress on finding a tour in $\calI$.\\
	\begin{theorem}
		Assume there exists an approximation algorithm, that given a vertebrate pair $(\calI, B)$, returns a tour of length at most $\kappa\cdot\val(\calI) + \eta\cdot w(B)$. Then there is a $\frac{\rho}{1 - 2\epsilon}$-approximation algorithm for irreducible instances, where $\rho = \kappa + \eta(\alpha_{NW} + 3)$
	\end{theorem}
	\begin{algorithm}\label{alg:6:irr-atsp}
		\caption{IRREDUCIBLE $\rightarrow$ VERTEBRATE PAIRS}
		Use \hyperref[lemm:6:q-bb]{Lemma \ref{lemm:6:q-bb}} to obtain a quasi-backbone $B$.\\
		$\lam^*_{\max} \leftarrow $ all non-singleton maximal sets in $\lam^*$.\\
		\For{each $S \in \lam^*_{\max}$} {
		Use recursion to find a tour $T_S$ in $\calI[S]$.\\
		$F_S\leftarrow$ lift of $T_S$.
		}
		$\calI'\leftarrow$ instance obtained by contracting all $S\in L_{max}^*$\\
		$T'\leftarrow$ tour in $\calI'$ obtained by running the vertebrate pairs algorithm on $(\calI', B)$.\\
		$T \leftarrow $ lift of $T'$.\\
		\Return  $\left(\cup\ F_S\right)\cup T$
	\end{algorithm}

	\begin{proof}
		The reduction is described in Algorithm 2. Note that step 8 is valid since all the sets in $L_{\geq 2}$ that were not visited by $B$ are actually singletons in $\calI'$.\vspace{2mm}
		\\That the total value of sets in $\lam^*_{\max}$ is at most $\epsilon\cdot \val(\calI)$ by definition of the quasi-backbone. So the total value of the recursive instances is $\sum_{S\in \lam^*_{\max}}2\cdot\val_{\calI}(S)\leqslant 2\epsilon\cdot\val(\calI)$. In other words, the total value at some recursive depth is a geometric sequence with respect to the depth whose ratio is equal to $2\cdot\epsilon$.\vspace{2mm}
		\\Since the non-recursive cost incurred by the algorithm is at most $(\kappa + \eta (\alpha_{NW} + 3))\cdot\val(\calI) = \rho$, and the $\val$ decays geometrically with recursion, the algorithm's cost is bounded by $\frac{\rho}{1-2\epsilon}\val(\calI)$.

	\end{proof}
\end{document}
