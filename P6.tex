\documentclass[./main.tex]{subfiles}

\DeclarePairedDelimiter\ceil{\lceil}{\rceil}
\DeclarePairedDelimiter\floor{\lfloor}{\rfloor}

\newcommand{\level}{\textsf{\textup{level}}}

\begin{document}
	\subsection{The Local Connectivity ATSP Problem}
		Fix some function $\lb : 2^V\rightarrow \mathbb{R}^+$ that has the following properties:
		\begin{enumerate}
			\item $\lb(S) = \sum\limits_{v\in S}\lb(\{v\})$ for all $S\subseteq V$
			\item $\lb(V) = \val(\calI)$
		\end{enumerate}
		One can think of \lb\ as being defined by distributing the LP value among the vertices in some specified way.\\

		\begin{problem}\
			\begin{itemize}
				\item[Input:]
					An instance $\calI$ along with a partition of the vertices into $V_1, \dots V_k$ such that each $V_i$ is strongly connected.
				\item[Output:] $F$, an eulerian set of edges such that for every $i$, $F$ exits (and hence enters) $V_i$ at least once.\\
			\end{itemize}
		\end{problem}

		\begin{definition}
			An algorithm for local connectivity ATSP is called $\bm{\alpha}\textbf{-light}$ if it outputs $F$ such that for each strongly connected component induced by $F$, say $\tilde{G}$, $w(\tilde{G})\leqslant \alpha\cdot \lb(\tilde{G})$.\\
		\end{definition}

		\begin{theorem}[Svensson '15]
			An $\alpha$-light algorithm for local connectivity ATSP implies a $O(\alpha)$-approximation algorithm for ATSP.
		\end{theorem}

		The paper describes an $O(1)$-light algorithm for local connectivity ATSP on vertebrate pairs, which gives us an $O(1)$-approximation algorithm for ATSP on vertebrate pairs, and hence an $O(1)$-approximation algorithm for general ATSP.\\
		We will also assume that $w(B)\leqslant O(1)\cdot \val(\calI)$, since this will be true for any vertebrate pair instance created by the previous reductions.\\

		\begin{definition}
			For this algorithm, we will define \lb\ as
			\[
				\textbf{\lb(v)}:=
				\begin{cases}
					\dfrac{\val(\calI)}{|V(B)|}\quad \text{if }v\in B \\
					2\cdot y_v\quad \text{otherwise}
				\end{cases}
			\]
		\end{definition}

		Note that this definition does not guarantee that $\lb(V) = \val(\calI)$, but $\lb(V)$ is always a constant factor away from $\val(\calI)$, so we can work with these \lb\ values and suffer a constant factor loss in the approximation ratio. Also, to make corner cases easier, we will add $V$ to $\lamNS$ for the rest of this write up.\vspace{2mm}\\
		The following is the main technical lemma of the paper. \vspace{1mm}
		\begin{lemma}[Lemma 7.3 in the paper]\label{lemma7}
			Given $U_1,\ldots,U_l$ disjoint subsets of $V\setminus V(B)$ that are strongly connected in $G$, such that for each $S\in \lamNS$, either $U_i\in S$ or $U_i$ is disjoint from $S$, we can find eulerian $F\subseteq E$ that guarantees the following:
			\begin{enumerate}[(1)]
				\item $w(F)\leqslant 3\cdot \val(I)$\label{lemm:main:1}

				\item $|\delta^-_F(U_i)|\geqslant 1$ for all $i$\label{lemm:main:2}

				\item $|\delta^-_F(v)|\leqslant 4$ for all $v$ such that $x(\delta^-(v)) = 1$\label{lemm:main:3}

				\item Any subtour in $F$ that crosses a set in $\lamNS$ visits a vertex of the backbone\label{lemm:main:4}
			\end{enumerate}
		\end{lemma}
		We will use the lemma to prove the following theorem, and prove the lemma later on.\\
		\begin{theorem}
			There is a $O(1)$-light algorithm for local connectivity ATSP on vertebrate pairs.
		\end{theorem}
		\begin{proof}
			We receive $(\calI,B)$ and $V_1,\ldots,V_k$ as input. We output $F^*=B\cup P\cup F$, where $P$ and $F$ are defined as:

			\begin{enumerate}
				\item[$P$:] If $B$ exists entirely within some $V_i$, then we pick $P$ the shortest cycle containing $u\in B$ and $v\notin V_i$. Otherwise we keep $P$ empty.

				\item[$F$:] WLOG, let $V_1,\ldots,V_l$ be the partitions that are disjoint from $B$. Now let $V_i'$ be the intersection of $V_i$ with a minimal set in $\lamNS$ that it intersects. Let $U_i$ be the first strongly connected component of $V_i'$ (in the topological order). Set $F$ to be the eulerian set guaranteed by running the algorithm of \hyperref[lemma7]{Lemma \ref{lemma7}} on $U_1,\ldots,U_l$.
			\end{enumerate}\vspace{2mm}

			Now we need to verify that $B\cup P\cup F$ is feasible, and that it is $O(1)$-light.\vspace{2mm}

			For each $V_i$, there are three possibilities:
			\begin{enumerate}
				\item[(a)] $V_i$ completely contains the backbone. The set $P$ guarantees that $F^*$ enters $V_i$ at least once.

				\item[(b)] $V_i$ partially contains the backbone. In this case, $B$ trivially crosses $V_i$, so $F^*$ must enter $V_i$.

				\item[(c)] $V_i$ is disjoint from the backbone. In this case, \hyperref[lemm:main:2]{(2)} tells us that some edge $(u,v)$ must enter $U_i$. $u\notin V_i'$ since $U_i$ is the source component of $V_i'$. So either $u\in S\setminus V_i$ or $u\notin S$. If the former is true, then $(u,v)$ enters $V_i$ from outside. If the latter is true, then $(u,v)$ crosses $S$, and \hyperref[lemm:main:4]{condition (4)} guarantees us that the subtour containing $(u,v)$ visits the backbone, which is outside $V_i$.
			\end{enumerate}

			Regardless, $F^*$ crosses each $V_i$, so it is indeed feasible.\vspace{2mm}

			Now consider any strongly connected component of $F^*$, say $\tilde{G}$. We would like to bound $w(\tilde(G))$ by $O(\lb(\tilde{G}))$. There are two possibilities for $\tilde{G}$:
			\begin{enumerate}
				\item[(a)] $\tilde{G}$ contains the backbone. So $$w(\tilde{G})\leqslant w(B) + w(P) + w(F)\leqslant O(\val(I)) + \val(\calI) + 3\cdot\val(\calI) = O(\val(\calI))$$
				      $$\lb(\tilde{G})\geqslant \lb(V(B)) = \val(\calI)$$
				      And it follows that $w(\tilde{G})\leqslant O(\lb(\tilde{G}))$.

				\item[(b)] $\tilde{G}$ is disjoint from the backbone. In this case, \hyperref[lemm:main:4]{(4)} guarantees that $\tilde{G}$ did not cross any set in $\lamNS$, so the only sets it might have crossed are singletons. Additionally, \hyperref[lemm:main:3]{(3)} guarantees that it crosses each singleton at most four times.
				\[
					w(\tilde{G})\leqslant \sum_{v\in V(\tilde{G})}4\cdot 2\cdot y_v = 4\cdot\lb(\tilde{G})
				\]
			\end{enumerate}
		\end{proof}

		\subsection{Proof of \hyperref[lemma7]{Lemma \ref{lemma7}}: The final ingredient}

			Order the laminar sets in $\lamNS$ by size so that $|S_1|\leqslant |S_2|\leqslant\cdots\leqslant |S_l| = |V|$. For each $v\in V$, define $\level(v)$ to be the index of the smallest set it is contained in.\\
			\begin{definition}
				An edge $e$ is called a \textbf{forward edge}, \textbf{backward edge}, or \textbf{neutral edge} if $e\in E_f$, $e\in E_b$, or $e\in E_n$ respectively, where $E_f, E_b,$ and $E_n$ are defined as:
				\begin{enumerate}[$-$]
					\item $\bm{E_f}:=\{(u,v)\in E\mid \level(u) > \level(v)\}$

					\item $\bm{E_b}:=\{u,v)\in E\mid \level(u) < \level(v)\}$

					\item $\bm{E_n}:=E\setminus (E_f\cup E_b)$\\
				\end{enumerate}
			\end{definition}

			\vspace{2mm}\pagebreak[2]
			\begin{definition}
				Call $\bm{G_{sp}}$ the \textbf{split graph}, where we define it as follows: for each $v\in V$, create $v^0$ and $v^1$ in $V(G_{sp})$. Create the following edges along with weights $\bm{w_{sp}}$:
				\begin{itemize}[$-$]
					\item $(v^0, v^1)$ for all $v\in V$ with weight $0$. Call these edges $0-1$ edges.

					\item $(v^1, v^0)$ for all $v\in V(B)$ with weight $0$. Call these edges $1-0$ edges.

					\item $(u^1, v^1)$ for all $(u,v)\in E_f\cup E_n$ with weight $w(u,v)$. Call these edges $1-1$ edges.

					\item $(u^0, v^0)$ for all $(u,v)\in E_b\cup E_n$ with weight $w(u,v)$. Call these edges $0-0$ edges.
				\end{itemize}
				Also for any set $S\subseteq V$, let $S^{sp}$ be its image in $G_{sp}$.
			\end{definition}

			Consider the lift of some subtour in $G_{sp}$ that crosses at least one set in $\lamNS$. Let $S$ be the smallest set in $\lamNS$ that the subtour crosses.
			Then the edge of the subtour going into $S$ must be a forward edge (i.e. a $1-1$ edge) by definition, and the edge going out of $S$ must be a backward edge (i.e. a $0-0$ edge). This means at some point, the subtour must use a $1-0$ edge, which it can only do if it crosses the backbone.
			\vspace{1mm}\\
			So if we restrict ourself to lifts of subtours of $G_{sp}$ (instead of all subtours of $G$), guarantee \hyperref[lemm:main:4]{(4)} will always be true.\vspace{2mm}

			\begin{lemma}\label{lemma6-1}
				We can find $x_{sp}$, a circulation on $G_{sp}$ so that the image of $x_{sp}$ in $G$ is $x$.
			\end{lemma}
			\begin{proof}
			We must have that for every edge $(u,v)$, $x_{sp}(u^0,v^0) + x_{sp}(u^1,v^1) = x(u,v)$. Note that for an edge that is forward or backward, exactly one of $(u^0,v^0)$ and $(u^1,v^1)$ exist, so the $x_{sp}$ values on these edges are alredy determined. So we just need to distribute $x_{sp}$ among the $0$- and $1$-copies of neutral edges.

			Let $f:E\rightarrow \mathbf{R}$ be $x_{sp}$ restricted to the $1$-copy of the graph. Then we would like to find $f$ with the following restrictions:
			\begin{enumerate}
				\item $0\leq f\leq x$.
				\item $f(e) = x(e)$ for forward edges $e$.
				\item $f(e) = 0$ for backward edges $e$.
				\item $f(\delta^+(v))\geq f(\delta^-(v))$ for all $v\notin V(B)$.
			\end{enumerate}
			Finding such an $f$ corresponds exactly to finding a valid $x_{sp}$. The fourth restriction capures the fact that for $v\notin V(B)$, there is a $v^0-v^1$ edge, but not a $v^1-v^0$ edge. So a net negative flow on the $1$-copy of a vertex can be fixed by routing it from the $0$-copy, but not the other way around.

			If we add an auxiliary vertex $s$ to $G$, and add $(s,u)$ edges for all $u\in V$ and $(s,u)$ edges for $u\in V(B)$, then (extending $f$ to these new edges) we can effectively turn restriction $4$ into the following restriction:
			\begin{enumerate}
				\item[$4$.] $f$ must be a circulation on the new graph, where $0\leq f(e)\leq \infty$ for all newly added edges $e$.
			\end{enumerate}
			In other words, we now have a circulation problem on $G\cup \{s\}$ with upper and lower bounds on each edge.
			\begin{theorem}[Hoffman's Circulation Theorem]
			Given a graph $(V,E)$, a circulation problem with upper and lower bounds on each edge (say $l$ and $u$ respectively) is feasible if and only if for each $S\subseteq V$, the following inequality holds:\begin{equation} \sum_{e\in \delta^-(S)}l(e)\ \leq \sum_{e\in \delta^+(S)}u(e)\label{equation:hoffman}\end{equation}
			\end{theorem}
			For convinience, we will use $l(\delta^-(S))$ and $u(\delta^+(S))$ for the left hand and right hand sides of \hyperref[equation:hoffman]{(\ref{equation:hoffman})} respectively.

			Fix some $S$. If $B\cap S\neq \emptyset$ or $\{s\}\cap S\neq \emptyset$, inequality \hyperref[equation:hoffman]{(\ref{equation:hoffman})} is trivially satisfied, as one of the new edges exits $S$, meaning that the right hand side of the inequality is $\infty$.

			So assume $B$ and $\{s\}$ are outside $S$. Note that we can ignore $s$, since it contributes nothing to either side of \hyperref[equation:hoffman]{(\ref{equation:hoffman})}. In some sense, $s$ has served its purpose by telling us which subsets of $V$ are important.

			We will show that $S$ obeys \hyperref[equation:hoffman]{(\ref{equation:hoffman})} using induction on the number of sets in $\lamNS$ that intersect with $S$. The base case is when every set in $\lamNS$ is disjoint from $S$. This case is trivial since the left hand side is zero.

			If at least one set in $\lamNS$ intersects $S$, let $T$ be such a set with the smallest index. Refer to \hyperref[figure:xsp1]{Figure \ref{figure:xsp1}} for the definitions of edge sets $A_s$, $A_l$, $C_b$, $C_f$, $D_S$, and $D_l$.
			% Also, let $A = A_s\cup A_l$ and $D = D_s\cup D_l$.
			Now, by the induction hypothesis, \hyperref[equation:hoffman]{(\ref{equation:hoffman})} must be true for $S':=S\setminus T$. In other words,
			\begin{align}
				l(\delta^-(S)) - l(D_l) - \underbrace{l(D_s)}_{=0} + \underbrace{l(C_b)}_{=0} = l(\delta^-(S')) &\leqslant u(\delta^+(S')) = u(\delta^+(S)) - u(A_l) - \underbrace{u(A_s)}_{= x(A_s)} + u(C_f)\nonumber\\
				\implies l(\delta^-(S)) - u(\delta^+(S)) \leqslant u(C_f) + l(D_l) - u(A_s) &\leqslant x(C_f) + x(D_l) - x(A_s) \label{P6:eqn:altproof}
			\end{align}
			Note that $l(C_b) = 0$ since all the edges in $C_b$ must be back edges by choice of $T$. similarly, $l(D_s) = 0$ as all edges in $D_s$ are back edges and, $u(A_s) = x(A_s)$ since all edges in $A_s$ must either be forward or neutral.
			If we want to prove $\hyperref[equation:hoffman]{(\ref{equation:hoffman})}$ for $S$, it is left to show that $x(C_f) + x(D_l) - x(A_s) \leqslant 0$.

			Since the backbone $B$ passes through $T$ and $B$ is disjoint from $S$, $T\setminus$ is nonempty. Note that since $T$ is laminar, $x(\delta^-(T)) = 1$.
			\[
				1\leq x(\delta^-(T\setminus S))=\underbrace{x(\delta^-(T))}_{=1} - x(C_f) - x(D_l) + x(A_s)
			\]

			So $x(C_f) + x(D_l) - x(A_s) \leqslant 0$. Using this in \hyperref[P6:eqn:altproof]{( \ref{P6:eqn:altproof})} we get, $l(\delta^-(S)) \leqslant u(\delta^+(S))$.

			% Note that $A_s$ cannot contain a back edge by the choice of $T$. So $u(A)\geq x(A_s)$. For the same reason, $D_s$ cannot have any forward edges, so $D_l$ alone contributes to $l(D)$. Finally, $C$ must only contain forward edges for the same reason, so $x(C) = u(C)$. Combining all these facts, we have \[l(D) + u(C)\leq x(C) + x(D_l)\leq x(A_s)\leq u(A)\]
			% This is what we had left to show.
			Therefore by induction, all $S$ obey \hyperref[equation:hoffman]{(\ref{equation:hoffman})}, and by Hoffman's Circulation Theorem, we can find a valid $f$. Hence we can find a valid $x_{sp}$, proving the lemma.
	\end{proof}
	\subfile{Diagrams/xsp1.tex}\vspace{5mm}

		At this point, it can be shown that (with a few modifications) rounding $x_{sp}$ to an integral solution and taking its image in $G$ would fulfil conditions $(1)$, $(3)$, and $(4)$ of \hyperref[lemma7]{Lemma \ref{lemma7}}. To fulfil $(2)$, the split graph is further modified taking into account the $U_i$s.\\
		\vspace{2mm}

		\begin{definition}\label{def:6.1}
			The modified split graph $\bm{G'_{sp}}$ and a circulation on it $\bm{x'_{sp}}$ are created by modifying $G_{sp}$ and $x_{sp}$ according to the following procedure for every $U_i$:
			\begin{itemize}[-]
				\item Add a new vertex called $a_i$.

				\item Pick $X_i^-$ to be a subset of incoming edges to $U_i^{sp}$ such that $x_{sp}(X_i^-) = 1/2$ and $X_i^-$ consists of either solely $1-1$ edges or solely $0-0$ edges. This must be possible since the total $x$-value coming into $U_i^{sp}$ is at least $1$.

				\item Let $X_i^+$ be a subset of edges leaving $U_i^{sp}$ that entered through some edge in $X_i^-$. This can be found using a path decomposition of $x_{sp}$ on $U_i$ and following each edge of $X_i^-$ along its paths until the paths leave $U_i$. Then $x_{sp}(X_i^+) = 1/2$.

				\item Redirect every edge in $X_i^-$ to $a_i$, and every edge in $X_i^+$ from $a_i$. In other words, $(u,v)\in X_i^-$ becomes $(u,a_i)$ and $(u,v)\in X_i^+$ becomes $(a_i, v)$. Also for each $u-v$ path we considered in the path decomposition (where $u\in X_i^-, v\in X_i^+$), reroute the $u-v$ flow in $x_{sp}$ flow through the edges $(u,a_i),(a_i,v)$. Call this modified flow $x'_{sp}$. Note that $x_{sp}'$ is still a circulation, and the original paths we considered are deleted from $G'_{sp}$.\\
			\end{itemize}
		\end{definition}
		We would like to round our circulation to an integral circulation.
		Note that for any vertex $v$ in $G_{sp}'$, we can easily modify the graph to impose capacity constraints for any subset $S$ of $\delta^-(v)$ by routing all the edges in $S$ to $v$ through a new intermediary vertex $\overline{v}$, and imposing a capacity constraint on the edge $(\overline{v},v)$.
		So we can impose the following capacity constraints for some circulation $z_{sp}'$:
		\begin{itemize}[-]
			\item For all vertices $v\in V$, let $S_0$ be the set of $0-0$ edges into $v^0$ and $S_1$ be the set of $1-1$ edges into $v^1$.
					Add the capacity constraints $z'_{sp}(S_0)\leqslant \ceil{2\cdot x_{sp}(S_0)}$ and $z'_{sp}(S_1)\leqslant \ceil{2\cdot x_{sp}(S_1)}$.

			\item For all vertices $a_i$, add the constraint $z'_{sp}(\delta^-(a_i)) = 1\  (\ = 2\cdot x'_{sp}(\delta^-(a_i))\ )$
		\end{itemize}
		Note that $2\cdot x_{sp}'$ is feasible with respect to these constraints by construction.
		So we can round $2\cdot x_{sp}'$ to obtain $z_{sp}'$, an integral, minimum-weight circulation which can be interpreted as an eulerian set $F'$.
		\vspace{2mm}\\
		We will use $F'$ to find an eulerian set of edges in $G_{sp}$. Note that for every $U_i$, $F'$ has an edge going into and out of $a_i$, say $(u,a_i),(a_i,v)$. Let their images in $G_{sp}$ be $(u,v'),(u',v)$ respectively. Then the image of $F'$ in $G_{sp}$ is not eulerian, since $v'$ and $u'$ have net incoming and outgoing flows respectively. We can fix this by simply adding a $v'-u'$ path.\\\vspace{2mm}

		\begin{claim}
			There is always a $v'-u'$ path in $G_{sp}$. Also, its weight is no more than $\val(U_i)$.\label{claim:7:uvpath}
		\end{claim}
		\begin{proof}
			First note that the assumptions in \hyperref[lemma7]{Lemma \ref{lemma7}} imply that all the edges within $U_i$ are neutral edges, so $G_{sp}$ contains both $0-0$ and $1-1$ copies of every edge in $U_i$. Also, since $U_i$ was strongly connected, the $0$-copy and $1$-copy of $U_i$ are strongly connected. Since there are $0-1$ edges corresponding to all vertices, there is always a $v'-u'$ path unless $v'$ is a $1$-copy and $u'$ is a $0$-copy. But $U_i$ is disjoint from the backbone, so by the construction of $G_{sp}$, it's not possible that $v'$ is a $1$-copy and $u'$ is a $0$-copy. By \hyperref[lemm:2:D-val]{Lemma \ref{lemm:2:D-val}}, the path weighs no more than $\val(U_i)$.\\
		\end{proof}
Now we can take the image of this eulerian set to $G$, say $F$, and guarantees \hyperref[lemm:main:2]{(2)} and \hyperref[lemm:main:4]{(4)} are automatically fulfilled.\vspace{2mm}

		\begin{claim}
			$w(F) \leqslant 3\cdot\val(\calI)$. That is, guarantee \hyperref[lemm:main:1]{(1)} holds.
		\end{claim}
		\begin{proof}
			\begin{align*}
				w(F) &\leqslant \sum_{e \in E_{sp}} w'_{sp}(e)z'_{sp}(e) + \sum_{i \in [l]} \val(U_i)\\
				 &\leqslant 2\sum_{e\in E_{sp}}w'_{sp}(e)x'_{sp}(e) + \val(\calI)\\
			 	&\leqslant 3\cdot\val(\calI)
			\end{align*}
		\end{proof}
		\begin{claim}
			If $\delta^-(v) = 1$, then $|\delta^-_F(v)|\leq 4$.
		\end{claim}
		\begin{proof}
			Fix some $v$ such that $\delta^-(v) = 1$, and let $S_0$ and $S_1$ be the set of incoming $0-0$ and $1-1$ edges into $v$ respectively. Then $1=x(\delta^-(v))\geq x'_{sp}(S_0) + x'_{sp}(S_1)$. Note that while converting $F'$ to $F$, we added a path in each $U_i$, so the in-degree of each vertex increased by at most $1$.
			\[
				|\delta^-_F(v)| \leqslant z'_{sp}(S_0) + z'_{sp}(S_1) + 1\leqslant \ceil{2x'_{sp}(S_0)} + \ceil{2x'_{sp}(S_1)} + 1
			\]
		\end{proof}

\end{document}
